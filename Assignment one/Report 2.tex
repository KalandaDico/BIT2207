\documentclass{article}

\begin{document}
\title{Report of the Computer Club on the IT Services in the school Computer Laboratory.}
\author{Kalanda Dico 215005312  15/U/5548/EVE}
\maketitle{}

\section{Introduction}
\subsection{}
As part of a wider review of the operation of the Computer Laboratory, the Head of Department established the computer club to review the provision of IT support services to staff and students. One purpose was to tackle issues that were the concern of both users and IT support staff. This report presents our findings and conclusions, and recommends possible courses of action to improve the comprehensive and high quality services that are currently provided.

\section{Procedure}
\subsection{}
We began our task by inviting written submissions giving information and views relevant to our terms of reference. The invitation, which gave the membership and terms of reference of the computer club, was sent to all staff, including those providing IT services, and research students. The Chairman of the Staff-Student Consultative Committee was asked to gather views from students.

\subsection{}
Written input was received from some 29 individuals, mainly in the form of handwritten messages. The comments received gave us a general picture of the state of IT services in the Laboratory from the viewpoints of both those providing them and those receiving them.

\subsection{}
We then arranged a series of interviews to discuss matters arising from the written submissions. As well as representatives of certain research groups and other specific areas of interest, we met Mr. Nyanzi Ronald, as head of the IT Support group, and had a separate meeting with the remaining computer staff.

\section{Overall View}
\subsection{}
Generally, we find that the computer staff are a highly skilled and hard-working group whose services are much appreciated within the Laboratory. Given the challenges of providing computer facilities and services in an academic environment, the Laboratory is one of the best served departments in the School. On the other hand, it would have been surprising not to have found any problems, and these can be summarised as the following issues of strategic importance:

\subsubsection{}
The computer staff team has been operating over many years and built up substantial skills. Given that technology continues to change, staff training is an important requirement. Similarly, the staff have developed good facilities, but these too require to be re-thought from time to time.

\subsubsection{}
The Laboratory has expanded greatly in recent years, which has created something of a policy vacuum. It is no longer sufficient for the policies of IT Support to be determined through the personal attention of the Head of the Laboratory, together with occasional consultation at the Wednesday meeting and somewhat irregular meetings of consultative groups. There are cases where commercial off-the-shelf solutions may be more appropriate than the expertly crafted tailor-made systems that are sometimes provided.

\subsubsection{}
There is a culture gap between the computer staff and some non-academic administrative staff. Ways of improving the support service provided to non-expert computer users need to be developed.

\section{Policy Making}
\subsection{}
There is currently no policy-making framework governing the provision of IT Support services. Many of our observations, conclusions and recommendations concern issues that can and should be resolved within such a framework.

\subsection{}
The following problems emerged during our investigations:

\subsubsection{}
The IT support team spend much time developing policies under the pressure of day-to-day operations, which creates a tendency for decisions to be technically driven rather than Laboratory driven. Computer staff do their best to balance competing arguments, but they are sometimes not best placed to decide things on their own.

\subsubsection{}
There are no clear rules as to what facilities, particularly personal facilities, should be provided to various categories of Laboratory member, be they academic staff, visitors, administrative staff, research students, etc. Nor is it clear who pays for what, and the extent to which anything is centrally funded; there is no obvious budget to support both the necessary capital and recurrent costs. It is also not clear the extent to which services should be funded centrally or charged directly to the client group or individual.

\subsubsection{}
The System Coordination Group, which was established some time ago to provide management for IT Support appears to have an idiosyncratic membership and is becoming dysfunctional.

\section{Conclusions}
\subsection{}
There is a need to establish an appropriate governance regime for IT Support so that strategic policy issues are considered and determined within the authority of the department as a whole. Such a mechanism must involve senior staff, and should provide appropriate consultative procedures to ensure that policy decisions reflect the needs of teaching, research and other support areas. Some kind of strategy committee is required, with a membership chosen to represent the Laboratory’s strategic interests.

\subsection{}
The same governance regime must be concerned with the financial aspects of IT services, deciding who should be provided with what and at whose expense. Services provided as a ‘free good’ must be budgeted for and controlled like any other basic facility, while charging and payment mechanisms may be appropriate in other circumstances.

\subsection{}
The Laboratory has to adopt various standards, deciding for example what basic IT facilities should be used and supported for various groups of staff and students. Within the constraints of general efficiency, these standards should recognise the different needs of different groups. The determination of such standards must be the business of the governance regime which should both be informed by user needs and advised by IT Support.

\subsection{}
The governance regime must also contain a managerial element; in particular, the head of IT Support must report to someone. We understand that the Laboratory is likely to establish a second Deputy Head of Department. On the assumption that the Head of Department will distribute various areas of responsibility between himself and his deputies, IT Support should be one of those areas. Either the Head or a Deputy Head should personally take responsibility for IT Support, and should both chair the strategy committee and be line manager to the Head of IT Support


\end{document}